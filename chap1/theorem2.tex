\section{Stabilitás kritériumok}

\begin{about}
  Mutassa be lineáris időinvariáns rendszerekre vonatkozóan az alapvető
  stabilitási definíciókat (BIBO stabilitás fogalma, magára hagyott rendszer
  stabilitása) és azok kritériumait (Routh-Hurwitz stabilitási kritérium, Bode
  stabilitási kritérium) szabályozási körökre alkalmazva!
\end{about}

A lineáris, időinvariáns rendszerekre jellemző, hogy érvényes rájuk a
szuperpozíció elve, valamint ugyanarra a bemenetre időponttól függetlenül
ugyanolyan választ adnak.

\subsection{BIBO stabilitás}

A BIBO (Bounded Input -- Bounded Output / Gerjesztés -- Válasz) stabilitási
feltétel azt vizsgálja, hogy véges bemenetre adott válasz véges kimenetet
eredményez-e. A bemenet korlátos, ha $\abs{u(t)} \leq A \le \infty$
egyenlőtlenség fennáll, valamint azt is tudjuk, hogy az operátor tartományban
végzett szorzás időtartományban konvolúciós integrálással ekvivalens.
Laplace-tartományban a kimenet a bemenet és a közöttük felírt átviteli függvény
szorzata, ezért $t$ tartományban a kimenet az alábbi alakban írható fel:
\begin{equation}
  y(t)
  = \int_0^\infty w(\tau) \cdot u(t - \tau) \, \mathrm d t.
  \label{eq:BIBO-conv}
\end{equation}
Majoráljuk a kimenet abszolút értékét az alábbi módon:
\begin{equation}
  \abs{y(t)}
  = \abs{\int_0^\infty w(\tau) \cdot u(t - \tau) \, \mathrm d t}
  \leq \int_0^\infty \abs{w(\tau)} \cdot \abs{u(t - \tau)} \, \mathrm d t
  \leq A \int_0^\infty \abs{w(\tau)} \, \mathrm d t.
  \label{eq:BIBO-major}
\end{equation}
Látható, hogy ha a rendszer $w(t)$ súlyfüggvénye lecsengő, állandósult értéke
zérus, akkor az alábbi improprius integrál konvergens, a kimenet ebből
következően véges lesz.

\subsection{Magára hagyott rendszer stabilitása}

\begin{figure}[htb]
  \centering
  \includestandalone{../graphics/asymptotic}
  \caption{
    A rendszer stabilitása valós és komplex pólusok esetén
    ({\color{darkBlue} stabil},
    {\color{darkYellow} határhelyzet},
    {\color{darkRed} instabil})
  }
  \label{fig:asymptotic-poles}
\end{figure}
Az aszimptotikus stabilitási feltétel (magára hagyott rendszer stabilitása)
azt vizsgálja meg, hogy a nyugalmi helyzetéből kimozdított rendszer visszatér-e
az egyensúlyi helyzetébe, ha a rendszert a kimozdítás után magára hagyjuk.

A szabályozott szakasz átviteli függvénye felírható az alábbi alakban:
\begin{equation}
  W_{cl}(s)
  = \frac{W_c(s) W_p(s)}{1 + W_c(s) W_p(s)}
  = \frac{W_o(s)}{1 + W_o(s)}
  = \frac{N_o / D_o}{1 + N_o / D_o}
  = \frac{N_o}{N_o + D_o}
  .
  \label{eq:asymptotic-cl}
\end{equation}
Készítsünk karakterisztikus polinomot, amely $N_o$ és $D_o$ polinomok
összegeként állítunk elő. Vizsgáljuk meg ezen polinom gyökeit. Megállapíthatjuk,
hogy amennyiben a gyökök valós részei negatívok, akkor a rendszer stabil lesz,
a kimenet állandósult értéke zérus.

\subsection{Routh-Hurwitz stabilitási kritérium}

A $p(s) = N_o + D_o$ karakterisztikus polinom gyökei bizonyos fokszám fölött
nehezen megtalálhatóak. Ilyen esetben alkalmazhatjuk a Routh-Hurwitz
kritériumot. Hozzuk a karakterisztikus polinomot az alábbi alakra:
\begin{equation}
  p(s) = a_n s^n + a_{n-1} s^{n-1} + \dots + a_1 s + a_0.
  \label{eq:RH-poles}
\end{equation}

A stabilitás szükséges feltétele, hogy a polinom együtthatói azonos előjelűek
legyenek, vagyis $\sgn a_0 = \sgn a_1 = ... = \sgn a_n$.

A stabilitás elégséges feltétele pedig, hogy a Hurwitz-mátrix minden egyes
főátlójára feszített aldeterminánsa azonos előjelű legyen. Ezen mátrix az
alábbi alakban írható fel:
\begin{equation}
  \rmat H = \begin{bmatrix}
    a_{n-1} & a_{n-3} & a_{n-5} & \hdots & 0      \\
    a_{n}   & a_{n-2} & a_{n-4} & \hdots & 0      \\
    0       & a_{n-1} & a_{n-3} & \hdots & 0      \\
    \vdots  & \vdots  & \vdots  & \ddots & \vdots \\
    0       & 0       & 0       & \hdots & a_0    \\
  \end{bmatrix}
  .
  \label{eq:RH-matrix}
\end{equation}

\subsection{Bode stabilitási kritérium}

\begin{figure}[htb]
  \centering
  \hspace{-4cm}
  \includestandalone{../graphics/bode-PM}
  \caption{Rendszer {\color{darkRed} fázis}- és {\color{darkYellow} erősítés}tartaléka}
  \label{fig:Bode-PM}
\end{figure}

A stabilitás vizsgálatot a Bode-diagram alapján is elvégezhetjük. Vizsgáljuk
a felnyitott kör átviteli függvényét. Éljünk az $s = \iu \omega$ formális
helyettesítéssel. Keressük meg $\omega_c$ vágási körfrekvenciát, ahol az
erősítés egységnyi ($\abs{W_o(\iu \omega_c)} = A(\omega_c) = 1 = 0 \,
  \mathrm{dB}$). Számítsuk ki a fázistartalékot az alábbi képlettel:
\begin{equation}
  \varphi_t = \varphi(\omega_c) - (- \pi)
  .
  \label{eq:Bode-PM}
\end{equation}
Amennyiben a fázistartalék pozitív, a rendszer stabil.
