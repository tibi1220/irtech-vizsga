\section{Párhuzamos PID szabályozó}

\begin{about}
  Párhuzamos arányos - integráló - deriváló (PID) szabályozó alkalmazása esetén
  ismertesse a pólus-zérus kiejtésen alapuló szabályozótervezés menetét!
  Válaszában térjen ki a másodrendű lengőtag szabályozásának esetére!
\end{about}

\subsection{A szabályozó felépítése}

\begin{figure}[htb]
  \centering
  \includestandalone{../graphics/PID}
  \caption{Párhuzamos PID szabályozó hatásvázlata}
  \label{fig:PID}
\end{figure}

A párhuzamos PID szabályozó egy arányos, egy integráló, és egy deriváló tag
összegeként áll elő. Egyenlete \textbf{ideális esetben}, operátor tartományban:
\begin{equation}
  W(s) = P + I / s + Ds.
  \label{eq:PID-s}
\end{equation}
A szabályozó egységugrásra adott válasza \textbf{ideális esetben},
időtartományban:
\begin{equation}
  v(t) = P + I t + D \dirac(t).
  \label{eq:PID-step}
\end{equation}

A valóságban ideális differenciáló tag nem létezik, tehát a PID szabályozó
átviteli függvénye \textbf{valóságos esetben}:
\begin{equation}
  W(s) = P + \frac{I}{s} + \frac{Ds}{1 + D's}.
  \label{eq:PID-real-s}
\end{equation}
\textbf{Megvalósítható} PID ugrás válasza pedig:
\begin{equation}
  v(t) = P + It + \left(\frac{D}{D'}\right) e^{\slashfrac{-t}{D'}}.
  \label{eq:PID-real-step}
\end{equation}

\subsection{Másodrendű lengőtag szabályozása}

A \textbf{másodfokú lengőtag} átviteli függvénye:
\begin{equation}
  W_p(s)
  = \frac{A \omega_n^2}{s^2 + 2 \zeta \omega_n s + \omega_n^2}
  = \frac{A}{1 + 2 \zeta T s + T^2 s^2}
  .
  \label{eq:W-ref}
\end{equation}

\textbf{Ideális PID szabályozó} egyenlete időállandós alakban
($T_I = P / I$ és $T_D = D / P$):
\begin{equation}
  W_c(s)
  = P \left( 1 + \frac{I}{P}\frac{1}{s} + \frac{D}{P}s \right)
  = P \left( 1 + \frac{1}{sT_I} + sT_D \right)
  = P \left( \frac{1 + sT_I + s^2 T_I T_D}{sT_I} \right)
  .
  \label{eq:PID-T}
\end{equation}
Ekkor a szabályozó $I$ és $D$ paraméterei:
\begin{equation}
  T_I  = 2 \zeta T
  , \quad \text{ és } \quad
  T_D = T / (2\zeta).
  \label{eq:PID-ID}
\end{equation}

\textbf{Megvalósítható PID} esetén is az előbb kiszámított időállandókat
alkalmazzuk, az extra időállandó pedig: $T_D' = nT_D$, ahol $n = 10$
(általában).
