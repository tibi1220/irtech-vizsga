\section{Soros PI - PD szabályozó}

\begin{about}
  Ismertesse a soros PI - PD szabályozó egyenletét idő és operátor tartományban!
  Mutassa be a pólus-zérus kiejtésen alapuló szabályozótervezés menetét!
  Válaszában térjen ki az ideális és a megvalósítható deriváló (D) tagok
  összefüggésére is!
\end{about}

% \subsection{A szabályozó felépítése}

\begin{figure}[htb]
  \centering
  \includestandalone{../graphics/PI-PD}
  \caption{Soros PI-PD szabályozó hatásvázlata}
  \label{fig:PIPD}
\end{figure}

Egy PI és egy PD szabályozó soros kapcsolásával egy soros PI-PD szabályozót
kapunk, melynek átviteli függvénye:
\begin{equation}
  W(s)
  =
  P_{PI} \left( \frac{1 + s\tau_i}{s\tau_i} \right)
  P_{PD} \left( \frac{1 + s\tau_d}{1 + s\tau_d'} \right)
  =
  \underbrace{P_{PI}P_{PD}}_{\kappa_c} \left(
  \frac{(1 + s\tau_i)(1 + s\tau_d)}{s\tau_i(1 + s\tau_d')}
  \right)
  .
  \label{eq:PIPD-1}
\end{equation}

% \subsection{Szabályozótervezés}

Kapcsolat a párhuzamos PID és a soros PI-PD szabályozók között:
\begin{equation}
  W(s)
  = \underbrace{
    \frac{\kappa_c}{\tau_i} \cdot
    \frac{1 + s (\tau_i + \tau_d) + s^2 (\tau_i \tau_d)}{s(1 + s\tau_d')}
  }_\text{Soros PI-PD}
  = \underbrace{
    \frac{P}{T_I} \cdot
    \frac{1 + s(T_I + T_D') + s^2(T_I(T_D + T_D'))}{s(1 + sT_D')}
  }_\text{Párhuzamos PID}
  .
  \label{eq:PIPD-2}
\end{equation}
Az egyenlőség akkor és csak akkor áll fenn, ha a törtek számlálója és nevezője
is megegyezik, valamint az erősítések is azonosak, vagyis:
\begin{alignat}{9}
   & \text{erősítés: } \quad
   &                                 & {\kappa_c}/{\tau_i} = {P}/{T_I}
  ,
  \\
   & \text{nevező: } \quad
   &                                 & \tau_d' = T_D'
  ,
  \\
   & \text{számláló ($s^1$): } \quad
   &                                 & \tau_i + \tau_d = T_I + T_D'
  ,
  \\
   & \text{számláló ($s^2$): } \quad
   &                                 & \tau_i \tau_d = T_I(T_D + T_D')
  .
\end{alignat}
A PID szabályozó paraméterei kifejezve a soros PI-PD szabályozó paramétereivel:
\begin{align}
  T_D' & = \tau_d
  ,
  \\
  T_I  & = \tau_i + \tau_d - \tau_d'
  ,
  \\
  P    & = (\kappa_c / \tau_i)(\tau_i + \tau_d - \tau_d')
  ,
  \\
  T_D  & = \frac{\tau_i \tau_d}{\tau_i + \tau_d - \tau_d'} - \tau_d'
  = \frac{
    \tau_i \tau_d - \tau_d'(\tau_i + \tau_d - \tau_d')
  }{\tau_i + \tau_d - \tau_d'}
  = \frac{(\tau_i - \tau_d')(\tau_d - \tau_d')}{\tau_i + \tau_d - \tau_d'}
  .
\end{align}
A tervezés lépéseit a következő, az ideális és megvalósítható deriváló tagok
összefüggéseit pedig az előző fejezetben tárgyaljuk.
