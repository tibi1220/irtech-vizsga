\section{Holtidős tag}

\begin{about}
  Ismertesse a holtidő (időkésés) fogalmát! Mutassa be a holtidős rendszerek
  esetén a szabályozó tervezésének menetét!
\end{about}

\subsection{Holtidős tag általános jellemzői}

\begin{figure}[htb]
  \centering
  \includestandalone{../graphics/time-delay}
  \caption{Ideális holtidős tag bemenete és kimenete}
  \label{fig:delay}
\end{figure}

A holtidő az az idő, ameddig a bemenet hatása nem érződik a kimeneten.

Ideális holtidős tag kimenetén a bemenet késleltetve jelenik meg, vagyis
$y(t) = u(t - \tau)$ egyenlőség áll fenn, ahol $\tau$ maga a holtidő. A kimenet
és bemenet közötti átviteli függvény:
\begin{equation}
  y(t) = u(t - \tau)
  \quad \rightarrow \quad
  Y(s) = e^{-\tau s} U(s)
  \quad \rightarrow \quad
  W_d(s) = e^{-\tau s}
  .
  \label{eq:delay-W}
\end{equation}

\subsection{Szabályozó tervezés holtidős rendszerek esetén}

\begin{figure}[htb]
  \centering
  \hspace{-4cm}
  \includestandalone{../graphics/bode-delay}
  \caption{
    Holtidős tag Bode Diagramja
    ($\tau=0,04 \, \mathrm s$,
    $\omega_\text{krit} = \pi / 0,04 \simeq 78,54 / \mathrm s$)
  }
  \label{fig:delay-bode}
\end{figure}

Képezzük a holtidős tag frekvencia átviteli függvényét. Éljünk az
$s = \iu \omega$ helyettesítéssel:
\begin{equation}
  W_d(\iu\omega) = e^{-\iu\omega\tau}
  \label{eq:delay-iuw}
\end{equation}
Megfigyelhetjük, hogy $e^{-\iu\omega\tau}$ komplex szám hossza egységnyi
($A(\omega) = 1$), valamint fázisszöge $\varphi(\omega) = -\omega\tau$.
A frekvencia növekedésével a fázistartalék egyre csökken, $\omega_\text{krit} =
  \pi / \tau$ frekvencia felett a fázistartalék negatív.

Tervezéskor először kiszámoljuk a késés nélküli rendszer vágási
körfrekvenciáját, majd ennek tudatában próbáljuk meg közelíteni a holtidővel
rendelkező rendszerét.
