\section{Kanonikus szabályozási kör felépítése}

\begin{about}
  Ismertesse a kanonikus szabályozási kör elvi felépítését, egy példán
  keresztül mutassa be az arányos (P), az integráló (I) és a deriváló (D) tagok
  szerepét a szabályozási körben!
\end{about}

\subsection{A szabályozási kör felépítése}

\begin{figure}[htb]
  \centering
  \includestandalone{../graphics/control_loop_text}
  \caption{A szabályozási kör elvi felépítése}
  \label{fig:control-loop-text}
\end{figure}

A szabályozási kör három szakaszból épül fel: a szabályozóból, a szabályozott
szakaszból, valamint a visszacsatoló tagból.
Egy ilyen rendszertől elvárjuk, hogy stabil és megvalósítható legyen.
A statikus pontosság (alapjelkövetés), zavar-kompenzáció és zajelnyomás is
fontos követelmény.

\begin{figure}[htb]
  \centering
  \includestandalone{../graphics/control_loop_basic}
  \caption{A szabályozási kör elvi felépítése}
  \label{fig:control-loop-basic}
\end{figure}

Tanulmányaink során SISO (Single Input Single Output / 1 bemenet 1 kimenet)
rendszerekkelfoglalkoztunk. A visszacsatoló hatást konstans egynek vettük, a
zajt és a zavarást pedig a legtöbb esetben elhanyagoltuk. A szabályozási kör
hatásvázlata ilyen esetben a \ref{fig:control-loop-basic}. ábrán látható.

\subsection{Arányos (P) tag hatása}

Alkalmazzunk arányos szabályozó tagot egy olyan szakaszon, melynek átviteli
függvénye egytárolós jellegű vagyis:
\begin{equation}
  W_c(s) = P,
  \quad \text{ és } \quad
  W_p(s) = \frac{A}{1 + sT}
  .
  \label{eq:P-use-CP}
\end{equation}
Ekkor a referenciajel és a kimenet közötti árviteli függvény:
\begin{equation}
  W_{cl}(s)
  = \frac{W_o}{1 + W_o}
  = \frac{PA / (1 + sT)}{1 + PA / (1 + sT)}
  = \frac{PA}{1 + sT + PA}
  = \frac{(PA) / (1 + PA)}{1 + (sT) / (1 + PA)}
  = \frac{\tilde A}{1 + s \tilde T}
  .
  \label{eq:P-use-cl}
\end{equation}

Az arányos taggal tehát gyorsítani tudjuk a rendszerünket, hiszen a jól
megválasztott $P$ taggal a rendszer időállandója csökkenthető. Ennek hatására
megváltozik viszont a rendszer erősítése is, maradó hibája lesz a kimenetnek,
méghozzá $e(\infty) = 1 - \tilde A$ nagyságú. Ezt a hibát statikus
alapjel-kompenzációval kiküszöbölhetjük. A referencia jelet egy $W_f(s) = 1 /
  \tilde A$ átviteli függvénnyel rendelkező arányos tagon kell átvezetni annak
érdekében, hogy a maradó hiba zérus legyen. A rendszer válasza ekkor
egységugrásra:
\begin{equation}
  v(t)
  = \frac{1}{\tilde A} \cdot \tilde A \left( 1 - e^{\slashfrac{-t}{T}} \right)
  = 1 - e^{\slashfrac{-t}{T}}
  \label{eq:P-use-step}
\end{equation}

\subsection{Integráló (I) tag hatása}

Alkalmazzunk most I szabályozót az előző alfejezetben bemutatott egytárolós
szakaszon. Ekkor az átviteli függvények:
\begin{equation}
  W_c(s) = \frac{I}{s},
  \quad \text{ és } \quad
  W_p(s) = \frac{A}{1 + sT}
  .
  \label{eq:I-use-CP}
\end{equation}
A referenciajel és a kimenet közötti árviteli függvény:
\begin{equation}
  W_{cl}(s)
  = \frac{AI}{s(1 + sT) + AI}
  = \frac{AI}{s^2 T + s + AI}
  = \frac{AI / T}{s ^ 2 + s / T + AI / T}
  = \frac{\omega_n^2}{s^2 + 2 \zeta \omega_n s + \omega_n^2}
  .
  \label{eq:I-use-cl}
\end{equation}

Látható, hogy a rendszer sajátkörfrekvenciája ($\omega_n = \sqrt{AI / T}$) és
csillapítása ($\zeta = 1 / (2 T \omega_n)$) az $I$ paraméter
megválasztásával beállítható. Ezek a rendszer további minőségi jellemzőit
is meg fogják határozni. A rendszer válasza egységugrásra:
\begin{equation}
  v(t)
  = 1 - e^{-\zeta \omega_n t} \left(
  \cos \omega_d t + \frac{\zeta}{\sqrt{1 - \zeta^2}}  \sin \omega_d t
  \right)
  .
  \label{eq:I-use-step}
\end{equation}

\subsection{Deriváló (D) tag hatása}

Alkalmazzunk most egy D tagot egy kéttárolós tag esetén. Ekkor az átviteli
függvények:
\begin{equation}
  W_c(s) = Ds,
  \quad \text{ és } \quad
  W_p(s) = \frac{A}{s (1 + sT)}
  .
  \label{eq:D-use-CP}
\end{equation}

Hatása hasonló a $P$ tag hatásáéhoz, itt is maradó hiba keletkezik, úgyhogy
önmagában nem alkalmazható.
