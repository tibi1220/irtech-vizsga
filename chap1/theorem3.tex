\section{Kanonikus szabályozási kör elvi felépítése}

\begin{about}
  Ismertesse a kanonikus szabályozási kör elvi felépítését, mutassa be az
  alapjel, a zavarás és a zaj hatását, mint a rendszer bemenetei a kimenő jelre,
  a hibajelre és a beavatkozó jelre, mint a rendszer kimenetei!
\end{about}

\begin{figure}[htb]
  \centering
  \includestandalone{../graphics/control_loop_disturbances}
  \caption{A szabályozási kör zavarásokkal, zajjal kiegészítve}
  \label{fig:control-loop-disturbances}
\end{figure}

\begin{multicols}{2}
  A szabályozási kör bemenetei:
  \begin{itemize}
    \item $R(s)$ -- referenciajel,
    \item $D_i(s)$ -- bemeneti zavarás,
    \item $D_o(s)$ -- kimeneti zavarás,
    \item $N(s)$ -- zaj.
  \end{itemize}

  A szabályozási kör kimenetei:
  \begin{itemize}
    \item $Y(s)$ -- kimenőjel,
    \item $E(s)$ -- hibajel,
    \item $U(s)$ -- beavatkozó jel.
  \end{itemize}
\end{multicols}

A bemenetek hatása a kimenőjelre:
\begin{equation}
  Y(s)
  = \left( \frac{W_c W_p} {1 + W_o} \right) R(s)
  + \left( \frac{W_p}     {1 + W_o} \right) D_i(s)
  + \left( \frac{1}       {1 + W_o} \right) D_o(s)
  + \left( \frac{-W_c W_p}{1 + W_o} \right) N(s)
  .
  \label{eq:disturbances-Y}
\end{equation}

A bemenetek hatása a beavatkozó jelre:
\begin{equation}
  U(s)
  = \left( \frac{W_c}     {1 + W_o} \right) R(s)
  + \left( \frac{-W_o}    {1 + W_o} \right) D_i(s)
  + \left( \frac{-W_c W_v}{1 + W_o} \right) D_o(s)
  + \left( \frac{-W_c}    {1 + W_o} \right) N(s)
  .
  \label{eq:disturbances-U}
\end{equation}

A bemenetek hatása a hibajelre:
\begin{equation}
  E(s)
  = \left( \frac{1}       {1 + W_o} \right) R(s)
  + \left( \frac{-W_p W_v}{1 + W_o} \right) D_i(s)
  + \left( \frac{-W_v}    {1 + W_o} \right) D_o(s)
  + \left( \frac{-1}      {1 + W_o} \right) N(s)
  .
  \label{eq:disturbances-E}
\end{equation}

A korábbi egyenletekben $W_o(s)$ a felnyitott kör átviteli függvénye, amely
az alábbi módon számítható:
\begin{equation}
  W_o(s) = W_c(s) W_p(s) W_v(s)
  .
  \label{eq:disturbances-o}
\end{equation}

Számításaink során gyakran egyszerűsítéseket teszünk. Elhanyagoljuk a kimeneti
zavarást, $W_v(s)$ függvény értékét pedig egységnyinek válasszuk meg.
Ebben az esetben a kimenet és a referenciajel között felírt átviteli függvény --
melyet gyakran a zárt szabályozási kör átviteli függvényének is nevezünk --
az alábbi alakra redukálódik:
\begin{equation}
  W_{cl}
  = \frac{W_c W_p}{1 + W_c W_p}
  = \frac{W_o}{1 + W_o}
  .
  \label{eq:disturbances-cl}
\end{equation}
