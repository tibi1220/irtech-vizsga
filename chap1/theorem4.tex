\section{PI és PD szabályozók}

\begin{about}
  Ismertesse a PI és PD szabályozó egyenletét idő és operátor tartományban!
  Mutassa be a pólus-zérus kiejtésen alapuló szabályozótervezés menetét!
  Válaszában térjen ki az ideális és a megvalósítható deriváló (D) tagok
  összefüggésére is!
\end{about}

\subsection{P, I és D tagok egyenletei}

% \subsubsection{P tag}

Az \textbf{arányos tag} operátor tartománybeli egyenlete: $W(s) = P$.
Idő tartománybeli ugrás válasza:
\begin{equation}
  v(t) = \mathcal{L}^{-1} \{ P / s \} = P.
  \label{eq:P-step}
\end{equation}

% \subsubsection{I tag}

Az \textbf{integráló tag} operátor tartománybeli egyenlete: $W(s) = I/s$.
Ugrás válasza:
\begin{equation}
  v(t) = \mathcal L^{-1} \{ I / s^2 \} = I t.
  \label{eq:I-step}
\end{equation}

% \subsubsection{Ideális D tag}

Az \textbf{ideális differenciáló} tag operátor tartománybeli egyenlete: $W(s) = Ds$.
Ugrás válasza:
\begin{equation}
  v(t) = \mathcal L^{-1} \{ D \} = D \dirac(t).
  \label{eq:D-step}
\end{equation}

% \subsubsection{Megvalósítható D tag}

Az ideális differenciáló tag nem megvalósítható, hiszen a számláló fokszáma
nagyobb a nevező fokszámánál, valamint végtelen kis idő alatt végtelen nagy
impulzust nem tudunk előállítani. Vezessünk be egy $D'$ paramétert, amelynek
értéke a gyakorlatban általában $D$ értékének tized része. Ekkor a
\textbf{megvalósítható differenciáló tag} átviteli függvénye:
\begin{equation}
  W(s) = \frac{Ds}{1 + D's}.
  \label{eq:D-real}
\end{equation}
A szabályozó ugrás válasza pedig:
\begin{equation}
  v(t)
  = \mathcal L^{-1} \left\{
  \frac{Ds}{1 + D's} \cdot \frac{1}{s}
  \right\}
  = \mathcal L^{-1} \left\{
  \frac{D}{D'} \frac{1}{s + 1 / D'}
  \right\}
  = \left( \frac{D}{D'} \right) e^{\slashfrac{-t}{D'}}
  \label{eq:D-real-step}
\end{equation}

\subsection{PI szabályozó}

A \textbf{PI szabályozó} egy arányos és egy integráló tag ősszegeként áll elő,
eredő átviteli függvénye:
\begin{equation}
  W(s) = P + \frac{I}{s}
  .
  \label{eq:PI-base}
\end{equation}
Irányítástechnikában szeretjük megadni a szabályozókat időállandós alakban,
hozzuk ilyen alakra $W(s)$ függvényt ($T_I = P/I$):
\begin{equation}
  W(s)
  = P \left( 1 + \frac{I}{sP}\right)
  = P \left( \frac{s + I/P}{s} \right)
  = P \left( \frac{1 + s (P/I)}{s (P/I)} \right)
  = P \left( \frac{1 + sT_I}{sT_I} \right)
  \label{eq:PI-time}
\end{equation}

\subsection{Ideális PD szabályozó}

Egy \textbf{ideális PD szabályozó} egy arányos és egy differenciáló tag
párhuzamos kapcsolásaként áll elő, átviteli függvénye:
\begin{equation}
  W(s) = P + Ds
  .
  \label{eq:PD-ideal-base}
\end{equation}
A megadás általában itt is időállandós alakban történik ($T_D = D/P$):
\begin{equation}
  W(s)
  = P + Ds
  = P \left( 1 + \frac{Ds}{P} \right)
  = P (1 + sT_D)
  .
  \label{eq:PD-ideal-time}
\end{equation}

\subsection{Megvalósítható PD szabályozó}

\textbf{Megvalósítható PD szabályozó} esetén valós differenciáló tagot
alkalmazunk. Így az eredő átviteli függvény:
\begin{equation}
  W(s)
  = P + \frac{Ds}{1 + D's}
  .
  \label{eq:PD-real-base}
\end{equation}
Ennek időállandós alakja:
\begin{equation}
  W(s)
  = P \left( 1 + \frac{D}{P} \frac{s}{1 + sT_D'} \right)
  = P \left( \frac{1 + s(T_D' + T_D)}{1 + sT_D'} \right)
  \approx P \left( \frac{1 + sT_D}{1 + sT_D'} \right)
  .
  \label{eq:PD-real-time}
\end{equation}

\subsection{Pólus-zérus kiejtéses szabályozótervezés}

Szabályozó tervezéskor a szabályozó pólusait és zérusait olyan értékűnek
érdemes beállítani, hogy azok kiejtsék a szabályozott szakasz nevezőjének
és számlálójának az általunk kívánt gyökeit. Fontos, hogy \textbf{csak stabil
  pólus vagy zérus ejthető ki}.

A tervezés során a szabályozott szakasz legnagyobb időállandóját szoktuk az
integráló, míg a második legnagyobb időállandót a differenciáló tag
időállandójának szoktuk választani, tehát $T_I = T_1$ és $T_D = T_2$.

Példa PID szabályozó esetén ($T_1 > T_2 > T_3$):
\begin{equation}
  W_p(s) = \frac{1}{(1 + sT_1)(1 + sT_2)(1 + sT_3)},
  \quad \text{ és } \quad
  W_c(s) = \frac{P (1 + sT_I)(1 + sT_D)}{sT_I(1 + nsT_D)}
  .
\end{equation}
Legyen $T_I = T_1$ és $T_D = T_2$, ekkor a felnyitott kör átviteli függvénye:
\begin{equation}
  W_o(s)
  = \frac{1}{(1 + sT_1)(1 + sT_2)(1 + sT_3)}
  \frac{P(1 + sT_1)(1 + sT_2)}{sT_1(1 + nsT_2)}
  = \frac{P}{sT_1(1 + nsT_2)(1 + sT_3)}
  .
\end{equation}
Az időállandókat már meghatároztuk, viszont $P$ tényező még mindig ismeretlen.
Ennek értékét beállíthatjuk olyanra, hogy egy számunkra kívánatos
fázistartalék jellemezze a rendszert. Ehhez először alakítsuk az átviteli
függvényünket frekvencia átviteli függvénnyé. Éljünk az $s = \iu \omega$
formális helyettesítéssel:
\begin{equation}
  W_o(\iu \omega)
  = \frac{P}{\iu \omega T_1(1 + n \iu \omega T_2)(1 + \iu \omega T_3)}
  .
  \label{eq:PID-iuw}
\end{equation}
Határozzuk meg $\omega_c$ vágási körfrekvenciát, ahol az erősítés egységnyi.
Ezen a frekvencián a fáziseltolás $-180^\circ$ és $\varphi_t$ fázistartalék
összege:
\begin{equation}
  \arg W(\iu \omega_c)
  = \varphi(\iu \omega_c)
  = -\pi + \varphi_t
  .
  \label{eq:PID-phi}
\end{equation}
A vágási körfrekvencia ismert, az arányos tag értéke az alábbi egyenlettel
számítható:
\begin{equation}
  \abs{W(\iu \omega_c)}
  = A(\iu \omega_c)
  = \frac{P}{
    \omega_c T_1
    \sqrt{1 + n^2 \omega_c^2 T_2^2}
    \sqrt{1 + \omega_c^2 T_3^2}
  } = 1
  .
  \label{eq:PID-A}
\end{equation}
Az egyenletet átrendezve:
\begin{equation}
  P = \omega_c T_1 \sqrt{1 + n^2 \omega_c^2 T_2^2} \sqrt{1 + \omega_c^2 T_3^2}
  .
  \label{eq:PID-P}
\end{equation}
A többi paraméter pedig:
\begin{equation}
  I = \frac{P}{T_I} = \frac{P}{T_1}
  ,\quad \text{ és } \quad
  D = P \cdot T_D = P \cdot T_2
  .
  \label{eq:PID-TD}
\end{equation}
