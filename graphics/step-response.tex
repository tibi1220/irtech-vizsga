\documentclass[border={-1.5cm, 0, 0, 0}]{standalone}

\usepackage{tikz}
\usepackage{amsmath}
\usepackage{unicode-math}
\usepackage{xcolor}
\colorlet{darkRed}{red!40!black}
\colorlet{darkBlue}{cyan!50!black}
\colorlet{darkYellow}{yellow!50!black}


\begin{document}

\usetikzlibrary{intersections}

\begin{tikzpicture}[thick, scale=1.2]
  \draw[-to] (-0.5,0) -- ++(6.0,0) node[below left] {$t$};
  \draw[-to] (0,-0.5) -- ++(0,4.5) node[below right] {$y(t)$};

  \def\on{3};         % omega_n
  \def\z{.3};         % zeta
  \def\od{2.8618176}; % omega_d = omega_n * sqrt(1 - zeta^2)
  \def\A{2.2};        % A -- gain
  \def\po{0.05};      % alpha
  \def\uval{2.42};    % A * (1 + alpha)
  \def\lval{1.98};    % A * (1 - alpha)

  \draw[darkRed, dashed] (0,\A) -- ++(5.25,0);
  \node[left] at (0, \A) {$y(\infty)$};
  \draw[draw=gray, dashed] (0,\uval) -- ++(5.25,0)
  node[above, pos=.82] {$y(\infty)(1 + \alpha\%)$};
  \draw[draw=gray, dashed] (0,\lval) -- ++(5.25,0)
  node[below, pos=.82] {$y(\infty)(1 - \alpha\%)$};

  \draw[
     draw =darkYellow,
    samples=20,
    domain=.55:5.2,
    dashed,
    smooth,
    variable=\x,
  ] plot ({\x}, {\A*(1+exp(-\x * \z * \on))});
  \draw[
     draw =darkYellow,
    samples=20,
    domain=0:5.2,
    dashed,
    smooth,
    variable=\x,
  ] plot ({\x}, {\A*(1-exp(-\x * \z * \on))});

  \draw[
     draw =darkBlue,
    very thick,
    samples=50,
    domain=0:5.2,
    smooth,
    variable=\x,
    name path=fn,
  ] plot ({\x}, {
      \A * (
      1 - exp(-\x * \z * \on) * (
      cos(deg(\x * \od))
      + \z*(1 - \z^2)^(-1/2) * sin(deg(\x * \od))
      )
      )
    });

  % Igen, ezek random próbálgatással készültek el xd
  \draw[dashed]
  (1.1,0) node[below] {$t_p$}
  |- (0,3.02) node[left] {$y_p$};
  \draw[dashed]
  (.66,0) node[below] {$t_r$}
  -- ++(0, \A);
  \draw[dashed]
  (2.46,0) node[below] {$t_r$}
  -- ++(0, \lval);
\end{tikzpicture}
\end{document}
