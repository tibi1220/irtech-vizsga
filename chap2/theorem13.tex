\section{Állapot megfigyelhetőség}

\begin{about}
  Folytonos idejű, lineáris, időinvariáns (LTI) rendszerek esetén mutassa be az
  állapot megfigyelhetőség definícióját, illetve a vizsgálathoz alkalmazott
  Kálmán-féle rangfeltételt! Vezesse le az állapot-megfigyelő tervezéséhez
  szükséges összefüggéseket hasonlósági transzformáció alkalmazása esetén!
  Válaszában térjen ki az Ackermann-formula alkalmazására is!
\end{about}

\subsection{LTI rendszerek állapot megfigyelhetősége}

\paragraph{Megfigyelhetőség}

Az $\rvec x_0$ kezdeti állapot megfigyelhető, ha az $\rvec u(t)$ és $\rvec y(t)$
$t \in (t_0; t_f)$ intervallumon vett értékei alapján meghatározható.

\paragraph{Rekonstruálhatóság}

Az $\rvec x(t)$ állapot előállítható $\rvec y(t)$ és $\rvec u(t)$ jelen- és
múltbeli állapotai alapján.

\paragraph{Kálmán-féle rangfeltétel}

Adott $n$ dimenziójú állapottér akkor megfigyelhető, hogy ha az $\rmat M_o$
megfigyelhetőségi mátrix rangja $n$, vagyis:
\begin{gather}
  \rank \rmat M_o = \dim \rmat A = n
  \quad - \quad \text{MIMO},
  \\
  \det \rmat M_o \neq 0
  \quad - \quad \text{SISO}.
\end{gather}

Az megfigyelhetőségi mátrix az alábbi alakban írható fel:
\bgroup
\def\arraystretch{1.2}
\begin{equation}
  \rmat M_o = \left[\begin{array}{ c }
      \rmat C         \\ \hdashline[3pt/3pt]
      \rmat C \rmat A \\ \hdashline[3pt/3pt]
      \vdots          \\ \hdashline[3pt/3pt]
      \rmat C \rmat A^{n-1}
    \end{array}\right]
  .
\end{equation}
\egroup

\subsection{Állapot-megfigyelő felépítése}

\begin{figure}[htb]
  \centering
  \includestandalone{../graphics/state-observer}
  \caption{Az állapot-megfigyelő hatásvázlata}
  \label{fig:state-observer}
\end{figure}

Állapot-megfigyelő tervezése során feltételezzük, hogy az összes állapot a
rendelkezésünkre áll. A célunk, hogy az összes állapot ismert legyen.

A rendszer és a megfigyelő egyenletei:
\begin{alignat}{9}
  \dot{\rvec x} & = \rmat A \rvec x + \rmat B \rvec u
                &
                & \text{állapotegyenlet,}
  \\
  \rvec y       & = \rmat C \rvec x
                &
                & \text{kimeneti egyenlet,}
  \\
  \dot{\hvec x} & = \rmat F \hvec x + \rmat L \rvec y + \rmat H \rvec u \quad \quad
                &
                & \text{megfigyelő egyenlete.}
\end{alignat}

Célunk, hogy a hiba állandósult állapota zérus legyen. A hiba felírható az
eredeti és a megfigyelt állapot különbségeként, vagyis:
\begin{equation}
  \rvec e = \rvec x - \hvec x
  \quad \rightarrow \quad
  \dot{\rvec e} = \dot{\rvec x} - \dot{\hvec x}
  = \rmat A \rvec x
  + \rmat B \rvec u
  - \rmat L \hvec x
  - \rmat L \rvec y
  - \rmat H \rvec u
  .
\end{equation}
Végezzük el a kiemeléseket, valamint használjuk az $\rvec y = \rmat C \rvec x$
kimeneti összefüggést:
\begin{equation}
  \dot{\rvec e}
  = (\rmat A - \rmat L \rmat C) \rvec x
  - \rmat F \hvec x
  + (\rmat B - \rmat H) \rvec u
  .
\end{equation}
A hiba deriváltja állandósult állapotban zérus, definiáljuk tehát az alábbi
változókat ezen gondolat szellemében (xd):
\begin{equation}
  \rmat F := \rmat A - \rmat L \rmat C,
  \quad \text{ és } \quad
  \rmat H := \rmat B.
\end{equation}
Ekkor az egyenletünk az alábbi alakra egyszerűsödik:
\begin{equation}
  \dot{\rvec e}
  = (\rvec A - \rvec L \rvec C) (\rvec x - \hvec x)
  = \rmat F \rvec e.
\end{equation}
A hibára felírt diffegyenlet megoldása idő tartományban:
\begin{equation}
  \rvec e(t)
  = \rvec e_0 \cdot e^{\rmat F t}.
\end{equation}
A rendszer tehát akkor lesz stabil, hogyha $\rmat F = \rmat A - \rmat L \rmat C$
mátrix sajátértékeinek valós részei negatívak.
% $\rmat F$ mátrix sajátértékei,
% ezáltal a rendszer pólusai is kisebbek, mint az állapot-visszacsato\-lás\-nál
% alkalmazott $\tmat A = \rmat A - \rmat B \rmat K$ mátrix sajátértékei.
% Itt az a célunk, hogy a hiba minél gyorsabban eltűnjön, minél hamarabb
% lehessen irányítani.

\subsection{Állapot-megfigyelő tervezése}

A tervezés lépései SISO rendszerek esetén:
\begin{enumerate}[label={\color{darkRed}\theenumi})]
  \setcounter{enumi}{-1}
  \item az megfigyelhetőség vizsgálata ($\det \rmat M_o \overset{?}{\neq} 0$),
  \item megfigyelhetőségi kanonikus alakra hozás (hasonlósági transzformációval),
  \item $\tmat L$ mátrix meghatározása az adott $\lambda_i$ sajátértékek
        segítségével ($\tmat A_c^\T = \tmat F = \tmat A - \tmat L \tmat C$),
  \item $\tmat L$ mátrix visszatranszformálása $\rmat L$ mátrixszá.
\end{enumerate}

Az irányíthatósági kanonikus alakra való jutáshoz hasonlóan itt is hasonlósági
transzformációt fogunk elvégezni. Vezessük be $\rmat T: \mathbb R^n \mapsto
  \mathbb R^n$ transzformációs mátrixot ($\rmat T \in \mathbb R^{n \times n}$,
invertálható), majd hattassuk ezen transzformációt az állapottérre:
\begin{alignat}{9}
  % noindent
	\dot{\tvec x} & = (\rmat T \rmat A \rmat T^{-1}) \tvec x + (\rmat T \rmat B) \rvec u
                && = \tmat A \tvec x + \tmat C \tvec u
	\\
	\rvec y       & = (\rmat C \rmat T^{-1}) \tvec x
                && = \tmat C \tvec x
  % indent
\end{alignat}

$\rmat T$ transzformációs mátrix meghatározható az megfigyelhetőségi mátrix
segítségével. Tudjuk hogy az eredeti az alábbi alakban írható fel:
\bgroup
\def\arraystretch{1.2}
\begin{equation}
  \rmat M_o = \left[\begin{array}{ c }
      \rmat C         \\ \hdashline[3pt/3pt]
      \rmat C \rmat A \\ \hdashline[3pt/3pt]
      \vdots          \\ \hdashline[3pt/3pt]
      \rmat C \rmat A^{n-1}
    \end{array}\right]
  .
\end{equation}
\egroup

A transzformált megfigyelhetőségi mátrix ezek alapján:
\bgroup
\def\arraystretch{1.2}
\begin{equation}
  \tmat M_o
  =
  \left[\begin{array}{ c }
      \tmat C         \\ \hdashline[3pt/3pt]
      \tmat C \tmat A \\ \hdashline[3pt/3pt]
      \vdots          \\ \hdashline[3pt/3pt]
      \tmat C \tmat A^{n-1}
    \end{array}\right]
  =
  \left[\begin{array}{ c }
      \rmat C \rmat T^{-1}                                \\ \hdashline[3pt/3pt]
      \rmat C \rmat T^{-1} (\rmat T \rmat A \rmat T^{-1}) \\ \hdashline[3pt/3pt]
      \vdots                                              \\ \hdashline[3pt/3pt]
      \rmat C \rmat T^{-1} (\rmat T \rmat A \rmat T^{-1})^{n-1}
    \end{array}\right]
  =
  \left[\begin{array}{ c }
      \rmat C         \\ \hdashline[3pt/3pt]
      \rmat C \rmat A \\ \hdashline[3pt/3pt]
      \vdots          \\ \hdashline[3pt/3pt]
      \rmat C \rmat A^{n-1}
    \end{array}\right] \rmat T^{-1}
  =
  \rmat M_o \rmat T^{-1}.
\end{equation}
\egroup

Ezek alapján a transzformációs mátrix az alábbi képlettel határozható meg:
\begin{equation}
  \rmat T = \tmat M_o^{-1} \rmat M_o.
\end{equation}

A megfigyelő egyenlete:
\begin{equation}
  \dot{\rvec e} = (\rmat A - \rmat L \rmat C) \rvec e
  \quad \overset{\rmat T}{\longrightarrow} \quad
  \dot{\tvec e} = (\tmat A - \tmat L \tmat C) \tvec e .
\end{equation}

Határozzuk meg a megfigyelést jellemző $\rmat L$ mátrixot a hasonlósági
transzformáció segítségével:
\begin{equation}
  (\tmat A - \tmat L \tmat C)
  = (\rmat T \rmat A \rmat T^{-1} - \tmat L \rmat C \rmat T^{-1})
  = (\rmat T \rmat A - \tmat L \rmat C) \rmat T^{-1}
  = \rmat T(\rmat A - \rmat T^{-1} \tmat L \rmat C) \rmat T^{-1}.
\end{equation}

Ezek alapján:
\begin{equation}
  \rmat L = \rmat T^{-1} \tmat L.
\end{equation}

\paragraph{Ackermann-formula}

$\rmat L$ mátrix az alábbi képlettel is meghatározható:
\begin{equation}
  \rmat L
  = \varphi(\rmat A)
  \;\rmat M_o^{-1}
  \;\uvec e_n
  .
\end{equation}
A képletben szereplő változók elnevezése:
\bgroup
\def\arraystretch{1.2}
\begin{center}
  \begin{tabular}{ | c c | }
    \hline
    $\uvec e_n$        & $n$-edik egységvektor               \\
    $\rmat M_c^{-1}$   & az irányíthatósági mátrix inverze   \\
    $\varphi(\rmat X)$ & elvárt pólusokat tartalmazó polinom \\
    \hline
  \end{tabular}
\end{center}
\egroup

\subsection{Kapcsolat az állapot-irányíthatóság és a megfigyelhetőség között}

Az Ackermann-formulák hasonlóak:

\begin{gather}
  \rmat K
  = \uvec e^\T_n
  \,\rmat M_c^{-1}
  \,\varphi(\rmat A)
  ,
  \\
  \rmat L
  = \varphi(\rmat A)
  \,\rmat M_o^{-1}
  \,\uvec e_n
  .
\end{gather}

A rendszer- kimeneti és bemeneti mátrixok közötti kapcsolat pedig:
\begin{gather}
  \rmat A_o = \rmat A_c^\T, \\
  \rmat B_o = \rmat C_c^\T, \\
  \rmat C_o = \rmat B_c^\T, \\
  \rmat D_o = \rmat D_c^\T.
\end{gather}

\clearpage
