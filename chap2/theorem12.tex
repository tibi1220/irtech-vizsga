\section{Állapot-visszacsatolás integráló taggal}

\begin{about}
  Folytonos idejű, lineáris, időinvariáns (LTI) rendszerek esetén ismertesse az
  integráló taggal kiegészített állapot-visszacsatolás tervezésének lépéseit!
  Válaszában térjen ki az Ac\-kermann-formula alkalmazására is!
\end{about}

\begin{figure}[htb]
  \centering
  \includestandalone{../graphics/state-feedback-I}
  \caption{Integráló tagos állapot-visszacsatolás hatásvázlata}
  \label{fig:I-feedback}
\end{figure}

Az előző fejezetben létre tudtunk hozni visszacsatolás-független
alapjelkompenzációt. Az alapjelet úgy formáltuk meg, hogy a maradó hiba
eltűnjön. Nem foglalkoztunk viszont a rendszerre ható zajokkal, illetve a
rendszer paraméter bizonytalanságaival. Egészítsük ki az
állapot-visszacsatolásunkat integráló taggal.

Vezessünk be egy új állapotot, amely a kimenet integráltja:
\begin{equation}
  x_I(t) = \int y(t) \, \mathrm d t
  \quad \rightarrow \quad
  y(t) = \dot x_I(t)
  .
\end{equation}

A kimeneti egyenlet ekkor ($K_I$ meghatározásához új pólust kell allokálni):
\bgroup
\def\arraystretch{1.2}
\begin{equation}
  u
  = K_I \underbracket{\int r(t) \, \mathrm d t}_{=0}
  - K_I \underbracket{\int y(t) \, \mathrm d t}_{=x_I(t)}
  - \rmat K \rvec x(t)
  =
  - \underbracket{\left[\begin{array}{c ;{2.5pt/2.5pt} c}
        \rmat K & K_I
      \end{array}\right]}_{=\tmat K} \left[\begin{array}{c}
      \rvec x(t)
      \\ \hdashline[2.5pt/2.5pt]
      x_I(t)
    \end{array}\right]
  = \tmat K \tvec x(t).
\end{equation}
\egroup

A bővített állapot-egyenlet:
\begin{gather}
  \dot{\tvec x}(t) = \tmat A \tvec x(t) + \tmat B u(t)
  ,
  \\
  \begin{bmatrix}
    \dot{\rvec x}(t) \\ \dot x_I(t)
  \end{bmatrix} = \begin{bmatrix}
    \rmat A & \rmat 0 \\ \rmat C & 0
  \end{bmatrix} \begin{bmatrix}
    \rvec x(t) \\ x_I(t)
  \end{bmatrix} \begin{bmatrix}
    \rmat B \\ 0
  \end{bmatrix} u(t)
  .
\end{gather}

A tervezés lépései:
\bgroup
\def\arraystretch{1.2}
\begin{enumerate}[label={\color{darkRed}\theenumi})]
  \item $\tmat A$ és $\tmat B$ mátrixok meghatározása,
  \item $\tmat M_c$ irányíthatósági mátrix meghatározása ($\dim \rmat A = n$,
        $\dim \tmat A = n+1$):
        \begin{equation}
          \tmat M_c = \left[\begin{array}{ c : c : c : c }
              \tmat B         &
              \tmat A \tmat B &
              \dots           &
              \tmat A^{n} \tmat B
            \end{array}\right],
        \end{equation}
  \item pólusallokálás, ezek segítségével $\varphi(s)$ karakterisztikus polinom
        felírása:
        \begin{equation}
          \varphi(s) = (s - \tilde p_1)(s - \tilde p_2) \dots (s - \tilde p_{n+1}),
        \end{equation}
  \item Az Ackermann-formula alkalmazása:
        \begin{equation}
          \tmat K = \left[\begin{array}{c c c c : c}
              k_1 & k_2 & \dots & k_n & k_I
            \end{array}\right]
          = \uvec e_n^\T
          \,\tmat M_c^{-1}
          \,\varphi({\tmat A})
        \end{equation}
\end{enumerate}
\egroup

\clearpage
