\section{Állapot megfigyelő}

\begin{about}
  Ismertesse az állapot-megfigyelőre alapozott állapot-visszacsatolás
  tervezésének lépéseit! Mutassa be az állapot-visszacsatolás tervezésének
  lépéseit egy bemenetű egy kimenetű (SISO) rendszerekre nézve Ackermann-formula
  segítségével! Válaszában térjen ki a szeparációs elvre!
\end{about}

\begin{figure}[htb]
  \centering
  \includestandalone{../graphics/state-observer-K}
  \caption{Megfigyelőre alapozott állapot-visszacsatolás hatásvázlata}
  \label{fig:observer-K}
\end{figure}

A állapotteret és a megfigyelőt jellemző egyenletek:
\begin{alignat}{9}
  \dot{\rvec x} & = \rmat A \rvec x + \rmat B \rvec u
                &
                & \text{állapotegyenlet,}
  \\
  \rvec y       & = \rmat C \rvec x
                &
                & \text{kimeneti egyenlet,}
  \\
  \dot{\hvec x} & = \rmat F \hvec x + \rmat L \rvec y + \rmat H \rvec u \quad \quad
                &
                & \text{megfigyelő egyenlete.}
\end{alignat}

$\rmat F = \rmat A - \rmat L \rmat C$ és $\rmat H = \rmat B$ egyenlőségek
továbbá is fennállnak, valamint a hiba is az előzőekhez hasonlóan a $\rvec e =
  \rvec x - \hvec x$ egyenlettel számítható. A hiba deriváltja ekkor:
\begin{equation}
  \dot{\rvec e}
  = \dot{\rvec x} - \dot{\hvec x}
  = (\rmat A - \rmat L \rmat C) \rvec e.
\end{equation}

A szabályozó egyenlete a hatásvázlat alapján:
\begin{equation}
  u
  = \tilde r - \rmat K \hvec x
  = \tilde r - \rmat K (\rvec e + \rvec x)
  = \tilde r - \begin{bmatrix}
    \rmat K & \rmat K
  \end{bmatrix} \begin{bmatrix}
    \rvec x \\ \rvec e
  \end{bmatrix}
\end{equation}

Írjuk fel az eredeti állapot és a hiba megváltozását hipermátrixos alakban:
\begin{gather}
  \begin{bmatrix}
    \dot{\rvec x} \\ \dot{\rvec e}
  \end{bmatrix} = \begin{bmatrix}
    \rmat A & \rmat 0 \\ \rmat 0 & \rmat F
  \end{bmatrix} \begin{bmatrix}
    \rvec x \\ \rvec e
  \end{bmatrix} + \begin{bmatrix}
    \rmat B \\ \rmat 0
  \end{bmatrix} u
  ,
  \\
  \begin{bmatrix}
    \dot{\rvec x} \\ \dot{\rvec e}
  \end{bmatrix} = \begin{bmatrix}
    \rmat A & \rmat 0 \\ \rmat 0 & \rmat F
  \end{bmatrix} \begin{bmatrix}
    \rvec x \\ \rvec e
  \end{bmatrix} + \begin{bmatrix}
    \rmat B \\ \rmat 0
  \end{bmatrix} \begin{bmatrix}
    \rmat K & \rmat K
  \end{bmatrix} \begin{bmatrix}
    \rvec x \\ \rvec e
  \end{bmatrix} + \begin{bmatrix}
    \rmat B \\ \rmat 0
  \end{bmatrix} \tilde r
  ,
  \\
  \begin{bmatrix}
    \dot{\rvec x} \\ \dot{\rvec e}
  \end{bmatrix} = \begin{bmatrix}
    \rmat A & \rmat 0 \\ \rmat 0 & \rmat F
  \end{bmatrix} \begin{bmatrix}
    \rvec x \\ \rvec e
  \end{bmatrix} + \begin{bmatrix}
    \rmat B \rmat K & \rmat B \rmat K \\ \rmat 0 & \rmat 0
  \end{bmatrix} \begin{bmatrix}
    \rvec x \\ \rvec e
  \end{bmatrix} + \begin{bmatrix}
    \rmat B \\ \rmat 0
  \end{bmatrix} \tilde r
  ,
  \\
  \begin{bmatrix}
    \dot{\rvec x} \\ \dot{\rvec e}
  \end{bmatrix} = \underbracket{\begin{bmatrix}
      \rmat A - \rmat B \rmat K &
      \rmat -\rmat B \rmat K      \\
      \rmat 0                   &
      \rmat A - \rmat L \rmat C
    \end{bmatrix}}_{\rmat{DZS}} \begin{bmatrix}
    \rvec x \\ \rvec e
  \end{bmatrix} + \begin{bmatrix}
    \rmat B \\ \rmat 0
  \end{bmatrix} \tilde r
  .
\end{gather}

A tervezés során kitétel hogy $\tilde r = 0$. Számítsuk ki $\rmat{DZS}$ mátrix
sajátértékeit a szeparációs elv segítségével:
\begin{equation}
  \tilde p(s)
  = \det(s \Imat - \rmat{DZS})
  = \det(s \Imat - \tmat A) \cdot \det(s \Imat - \rmat F)
  .
\end{equation}

\bgroup
\samepage
A megfigyelőt és a visszacsatolást külön tervezzük, melyek lépései:
\begin{enumerate}[label={\color{darkRed}\theenumi})]
  \item Az irányíthatóság vizsgálata ($\det \rmat M_c \overset{?}{\neq} 0$):
        \begin{equation}
          \rmat M_c = \begin{bmatrix}
            \rmat B         &
            \rmat A \rmat B &
            \dots
            \rmat A^{n-1} \rmat B
            .
          \end{bmatrix}
        \end{equation}
  \item Pólusallokáció a visszacsatolás számára:
        \begin{gather}
          p_{12}
          = -\beta \pm \iu \omega_d,
          \\
          \varphi_c(s)
          = (s - p_1)(s - p_2)
          = s^2 + 2 \beta s + (\beta^2 + \omega_d^2).
        \end{gather}
  \item $\rmat K$ mátrix meghatározása az Ackermann-formula segítségével:
        \begin{equation}
          \rmat K
          = \uvec e_n^\T
          \,\rmat M_c^{-1}
          \,\varphi_c(\rmat A)
          .
        \end{equation}
  \item A megfigyelhetőség vizsgálata ($\det \rmat M_o \overset{?}{\neq} 0$):
        \begin{equation}
          \rmat M_o = \begin{bmatrix}
            \rmat C         &
            \rmat C \rmat A &
            \dots
            \rmat C \rmat A^{n-1}
          \end{bmatrix}^\T
          .
        \end{equation}
  \item Pólusallokáció a visszacsatolás számára
        (multiplikáció megengedett, $c > 1$, tetszőleges):
        \begin{gather}
          q_{12}
          = -c \beta
          \\
          \varphi_c(s)
          = (s - q_1)(s - q_2)
          = s^2 + 2c \beta s + c^2 \beta^2
        \end{gather}
  \item $\rmat K$ mátrix meghatározása az Ackermann-formula segítségével:
        \begin{equation}
          \rmat L
          = \varphi_o(\rmat A)
          \,\rmat M_o^{-1}
          \,\uvec e_n
          .
        \end{equation}
\end{enumerate}
\egroup

Fontos, hogy tervezés során a visszacsatoláshoz választott gyökök valós részei
nagyobbak legyenek (kevésbé legyenek negatívak), mint a megfigyeléshez
allokált gyökök valós részéi. Ekkor a hiba gyorsan el fog tűnni.

\begin{figure}[htb]
  \centering
  \begin{tikzpicture}[thick]
    \draw[-to] (-3,0) -- (2,0) node[above left] {$\Re$};
    \draw[-to] (0,-2) -- (0,2) node[below right] {$\Im$};

    \draw[darkYellow, dashed] (-1.5, -1.65) -- ++(0, 3.3);
    % Igen ezek szorzásjelek xd
    \node[darkRed] at (-.75, 0) {$\times$};
    \node[darkRed] at (-.5, 1) {$\times$};
    \node[darkRed] at (-.5,-1) {$\times$};
    \node[darkRed] at (-1.25, .8) {$\times$};
    \node[darkRed] at (-1.25,-.8) {$\times$};

    \draw[decorate, decoration={brace}, draw=darkRed]
    (-.05,-1.6) -- (-1.45,-1.6) node[midway, below=1mm] {$\tmat A$};

    \node[darkBlue] at (-2.65, -1) {$\times$};
    \node[darkBlue] at (-2.65, 1) {$\times$};
    \node[darkBlue] at (-2.25,0) {$\times$};
    \node[darkBlue] at (-1.9, .6) {$\times$};
    \node[darkBlue] at (-1.9,-.6) {$\times$};

    \draw[decorate, decoration={brace}, draw=darkBlue]
    (-1.55,-1.6) -- (-2.95,-1.6) node[midway, below=1mm] {$\rmat F$};
  \end{tikzpicture}
  \caption{A rendszer pólusainak elhelyezkedése a komplex számsíkon}
  \label{fig:poles}
\end{figure}
