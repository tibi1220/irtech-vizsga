\section{Minőségi jellemzők}

\begin{about}
  Mutassa be a szabályozási körök minőségi jellemzőjének definícióit! Ismertesse
  a minőségi követelmények alapján történő pólusválasztás módszerét! Válaszában
  térjen ki a lineáris idő invariáns (LTI) rendszerek esetén az állapottér
  egyenletek rendszermátrixának sajátértékei és a rendszer átviteli függvényének
  pólusai közötti kapcsolatra is!
\end{about}

\subsection{A szabályozási körök minőségi jellemzői}

\begin{figure}[htb]
  \centering
  \includestandalone{../graphics/step-response}
  \caption{Kéttárolós lengő tag ugrás válasza}
  \label{fig:step}
\end{figure}

Egy kéttárolós lengő tag átviteli függvénye:
\begin{equation}
  W(s) = \frac{A \omega_n^2}{s^2 + 2 \zeta \omega_n s + \omega_n^2}
  .
  \label{eq:W-2-ref}
\end{equation}

Ennek a tagnak az ugrás válasza:
\begin{equation}
  v(t) = A \left(
  1 - e^{-\zeta \omega_n t} \left(
  \cos \omega_d t
  + \frac{\zeta}{\sqrt{1 - \zeta^2}} \sin \omega_d t
  \right)
  \right)
  .
  \label{eq:W-2-ref-step}
\end{equation}

Az egyenletben szereplő változók elnevezése:
\begin{center}
  \begin{tabular}{ | c c | }
    \hline
    $A$        & {Erősítés}                          \\
    $\omega_n$ & {Csillapítatlan sajátkörfrekvencia} \\
    $\zeta$    & {Csillapítási tényező}              \\
    $\omega_d$ & {Csillapított sajátkörfrekvencia}   \\
    \hline
  \end{tabular}
\end{center}

\paragraph{Stabilitás}

Egy rendszer stabil, ha a rendszermátrix sajátértékeinek valós része negatív.
Egy rendszer pólusai között mindig páros számú komplex gyök található.
Másodrendű rendszer esetén ezek az alábbi alakban írhatóak fel (amennyiben
$\zeta < 1$):
\begin{equation}
  p_{12}
  = -\zeta \omega_n \pm \iu \omega_n \sqrt{1 - \zeta^2}
  = -\beta \pm \iu \omega_d.
  \label{eq:complex-poles}
\end{equation}

\paragraph{Felfutási idő}

Azt az időpillanatot, amikor a válaszfüggvény először felveszi az állandósult
állapotot, mint függvényértéket, felfutási időnek ($t_r$ -- rise time) nevezzük.

\paragraph{Beállási idő}

Azt az időpillanatot, amelytől fogva a válaszfüggvény az állandósult érték
$\alpha\%$ sugarú környezetében marad, és onnan már nem lép ki, beállási időnek
($t_s$ -- settling time, vagy $T_{\alpha\%}$) nevezzük.

\paragraph{Csúcsidő}

A csúcsértékhez ($y_p$) tartozó időpillanatot csúcsidőnek ($t_p$ -- peak time)
nevezzük.

\paragraph{Százalékos túllövés}

A százalékos túllövés ($\Delta v$, vagy $\mathit{PO}$) megmutatja, hogy a
csúcsérték hány százalékban tér el az állandósult értéktől:
\begin{equation}
  \Delta v = \frac{y_p - y(\infty)}{y(\infty)} \cdot 100\%.
  \label{eq:PO}
\end{equation}

\paragraph{Pólusválasztás} A rendszer pólusait $p_{12} = -\beta \pm \iu
  \omega_d$ alakban keressük. A pólusok helyes megválasztásával beállíthatjuk
a beállási idejét, valamint százalékos túllövését. Ezek az alábbi képletekkel
számíthatóak:
\begin{alignat}{9}
  T_{\alpha\%}
   & = \frac{1}{\beta} \cdot \ln \left( \frac{100}{\alpha} \right)
   & = \frac{1}{\zeta \omega_n} \cdot \ln \left( \frac{100}{\alpha} \right),
  \\[2mm]
  \Delta v
   & = \exp \left( \frac{-\pi \beta}{\omega_{d}} \right)
   & = \exp \left( \frac{-\pi \zeta}{\sqrt{1 - \zeta^2}} \right).
\end{alignat}

A számításokat egyszerűbb a $\beta$ és $\omega_d$ változókra elvégezni, és
később ezekből visszaszámolni $\zeta$ és $\omega_n$ paramétereket:
\begin{gather}
  \beta = \zeta \omega_n,
  \\
  \omega_d = \omega_n \sqrt{1 - \zeta^2}.
\end{gather}

\subsection{A rendszermátrix és a pólusok közötti kapcsolat}

Az állapottér modell segítségével felírható a rendszer átviteli mátrixa
(kezdeti feltétel: $\rvec x_0 = \nvec$):
\begin{alignat}{9}
   & \dot{\rvec x} (t) & = & \rmat A \rvec x(t) & + & \rmat B \rvec u(t)
  \quad \overset{\mathcal L}{\longrightarrow} \quad
   & s \rvec X(s)      & = & \rmat A \rvec X(s) & + & \rmat B \rvec U(s)
  ,
  \\
   & \rvec y(t)        & = & \rmat C \rvec x(t) & + & \rmat D \rvec u(t)
  \quad \overset{\mathcal L}{\longrightarrow} \quad
   & \rvec Y(s)        & = & \rmat C \rvec X(s) & + & \rmat D \rvec U(s)
  .
\end{alignat}

Az állapotegyenlet átrendezve:
\begin{equation}
  \rvec X(s) = (s\Imat - \rmat A)^{-1} \, \rmat B \rvec U(s)
  \quad \rightarrow \quad
  \rvec Y(s)
  = \rmat C \, (s\Imat - \rmat A)^{-1} \, \rmat B \rvec U(s)
  + \rmat D \rvec U(s)
  .
  \label{eq:SSM-S1}
\end{equation}

Az átviteli mátrix a ki- és bemenet hányadosa, vagyis:
\begin{equation}
  \rmat W(s)
  = \rmat C \, (s\Imat  - \rmat A)^{-1} \, \rmat B + \rmat D
  = \frac{
    \rmat C \, \adj(s\Imat  - \rmat A) \, \rmat B
    + \det(s\Imat - \rmat A) \, \rmat D
  }{\det (s\Imat - \rmat A)}
  .
  \label{eq:SSM-W}
\end{equation}

Láthatjuk, hogy az előző egyenlet nevezőjének gyökei $\rmat A$ mátrix
sajátértékei, vagyis az átviteli függvény pólusai megegyeznek a rendszermátrix
sajátértékeivel.
