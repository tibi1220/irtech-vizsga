\section{Alapjelkompenzáció}

\begin{about}
  Folytonos idejű, lineáris, időinvariáns (LTI) rendszerek esetén ismertesse
  az alapjelkompenzációval kiegészített állapot-visszacsatolás tervezésének
  lépéseit! Vezesse le az állapot-visszacsatolástól függő és független
  alapjelkompenzáció meghatározására szolgáló összefüggéseket!
\end{about}

Állapot-visszacsatolás esetén a rendszer maradó hibája nem zérus, érdemes
alapjelkompenzációt alkalmazni, hogy ne legyen a rendszernek hibája. A
kompenzáció lehet állapot-visszacsato\-lástól függő és független is.

\subsection{Állapot-visszacsatolástól függő alapjelkompenzáció}

\begin{figure}[htb]
  \centering
  \includestandalone{../graphics/state-feedback-K}
  \caption{Visszacsatolástól függő alapjelkompenzáció hatásvázlata}
  \label{fig:K-feedback}
\end{figure}

Először írjuk fel az állapottér-modell egyenleteit, illetve a szabályozási
törvényt:
\begin{alignat}{9}
  \dot{\rvec x}(t) & = \rmat A \rvec x(t) + \rmat B \rvec u(t),
  \\
  \rvec y(t)       & = \rmat C \rvec x(t),
  \\
  \rvec u(t)       & = \rvec r(t) - \rmat K \rvec x(t).
\end{alignat}
Az állapot-egyenletbe helyettesítsük be a szabályozási törvényt:
\begin{equation}
  \dot{\rvec x}
  = (\rmat A - \rmat B \rmat K)\rvec x(t) + \rmat B \rvec r(t)
  = \tmat A \rvec x(t) + \rmat B \rvec r(t).
\end{equation}
Térjünk át operátor tartományba, és fejezzük ki a kimenetet:
\begin{equation}
  (s \Imat - \tmat A) \rvec X(s) = \rmat B \rvec R(s)
  \quad \rightarrow \quad
  \rvec Y(s) = \rmat C \, (s \Imat - \tmat A)^{-1} \, \rmat B \rvec R(s).
\end{equation}
Az állandósult állapot meghatározásához alkalmazzuk a végérték-tételt
($r(t) = \step (t)$):
\begin{equation}
  \rvec y(\infty)
  = \lim_{s \rightarrow 0} s \rvec Y(s)
  = \lim_{s \rightarrow 0} s \rmat C \, (s \Imat - \tmat A)^{-1} \, \rmat B \, \frac{\rvec r_0}{s}
  = \lim_{s \rightarrow 0} \rmat C \, (s \Imat - \tmat A)^{-1} \, \rmat B \rvec r_0
  = - \rmat C \tmat A^{-1} \rmat B \, \rvec r_0.
\end{equation}
Látható, hogy a kimenet állandósult értéke a referenciajel konstans szorosa
lesz. Ha az alapjelet formáljuk, és ezzel az ún. erősítéssel leosztjuk értékét,
akkor a kimenet értéke meg fog egyezni a referenciajel értékével, a maradó hiba
eltűnik:
\begin{equation}
  \rmat K_r = -(\rmat C \tmat A^{-1} \rmat B)^{-1}.
\end{equation}

Megfigyelhetjük, hogy $\rmat K_r$ függ a visszacsatolástól. Ez azt jelenti,
hogy amennyiben meg szeretnénk változtatni a rendszer allokált pólusait,
$\rmat K_r$ értékét minden egyes alkalommal újra kell számítani.

\subsection{Állapot-visszacsatolástól független alapjelkompenzáció}

\begin{figure}[H]
  \centering
  \includestandalone{../graphics/state-feedback-N}
  \caption{Visszacsatolástól független alapjelkompenzáció hatásvázlata}
  \label{fig:N-feedback}
\end{figure}

A visszacsatolás független alapjelkompenzáció feloldja az előbb említett
problémát. Bontsuk fel a $\rmat K_r$ mátrixunkat az alábbi módon:
\begin{equation}
  \rmat K_r(\rmat K)
  = \rmat K_{ru} + \rmat K \rmat K_{rx}
  = \rmat N_u + \rmat K \rmat N_x.
\end{equation}
Ekkor a szabályozó egyenlete az alábbi alakot veszi fel:
\begin{equation}
  \rvec u(t)
  = \rmat K_r \rvec r(t) - \rmat K \rvec x(t)
  = (\rmat K_{ru} + \rmat K \rmat K_{rx}) \rvec r(t) - \rmat K \rvec x(t)
  = \rmat K_{ru} \rvec r(t) + \rmat K (\rmat K_{rx} \rvec r(t) - \rvec x(t))
  .
\end{equation}
A cél itt is a maradó hiba elkerülése, vagyis hogy a kimenet állandósult értéke
megegyezzen a referenciajel értékével. Írjuk fel a rendszert leíró egyenleteket:
\begin{alignat}{9}
  % noindent
  \dot{\rvec x}(t) & = \rmat A \rvec x(t) + \rmat B \rvec u(t)
                   & \quad\rightarrow\quad 
                   && \nvec 
                   &= \rmat A \rvec x(\infty) + \rmat B \rvec u(\infty)
                   ,
  \\
  \rvec y(t)       & = \rmat C \rvec x(t)
                   & \quad\rightarrow\quad 
                   && \rvec y(\infty) 
                   &= \rmat C \rvec x(\infty) = \rvec r_0
                   ,
  \\
  \rvec u(t)       & = \rmat K_{ru} \rvec r(t) + \rmat K (\rmat K_{rx} \rvec r(t) - \rvec x(t))
                   & \quad\rightarrow\quad 
                   && \rvec u(\infty) 
                   &= \rmat K_{ru} \rvec r_0 + \rmat K (\rmat K_{rx} \rvec r_0 - \rvec x(\infty))
                   .
  % indent
\end{alignat}
Az állandósult állapotban $\rvec x(\infty) = \rmat K_{rx} \rvec r_0$ egyenlőség
fennáll, hiszen nincs maradó hibánk. Helyettesítsük be a szabályozó egyenletét
az állapot-egyenletbe:
\begin{equation}
  \nvec
  = \rmat A \rvec x(\infty) + \rmat B \rvec u(\infty)
  = \rmat A \rmat K_{rx} \rvec r_0 + \rmat B \rmat K_{ru} \rvec r_0
  .
\end{equation}
A kimeneti egyenletből pedig következik, hogy:
\begin{equation}
  \rvec r_0 = \rmat C \rvec x(\infty) = \rmat C \rmat K_{rx} \rvec r_0
  \quad \rightarrow \quad
  \rmat C \rmat K_{rx} = \Imat.
\end{equation}
Az előző két egyenletet felírhatjuk mátrixos alakban is:
\bgroup
\ADLdrawingmode{1}
\def\arraystretch{1.2}
\begin{equation}
  \left[\begin{array}{ c : c }
      \rmat A & \phantom{\rmat A} \hspace{-2.8mm}\rmat B
      \\ \hdashline
      \rmat C & \phantom{\rmat A} \hspace{-2.5mm}\rmat 0
    \end{array}\right]
  \left[\begin{array}{ c }
      \rmat K_{rx}
      \\ \hdashline
      \rmat K_{ru}
    \end{array}\right]
  =
  \left[\begin{array}{ c }
      \rmat 0
      \\ \hdashline[2.25pt/2.25pt]
      \Imat
    \end{array}\right]
  .
\end{equation}
\egroup
Aki szeret szenvedni, az a mindenkori matematikai átalakítások segítségével
invertálhatja a rendszermátrixot tartalmazó hipermátrixot:
\bgroup
\ADLdrawingmode{1}
\def\arraystretch{1.2}
\begin{equation}
  \left[\begin{array}{ c }
      \rmat K_{rx}
      \\ \hdashline
      \rmat K_{ru}
    \end{array}\right]
  =
  \left[\begin{array}{ c : c }
      \rmat A & \phantom{\rmat A} \hspace{-2.8mm}\rmat B
      \\ \hdashline
      \rmat C & \phantom{\rmat A} \hspace{-2.5mm}\rmat 0
    \end{array}\right]^{-1}
  \left[\begin{array}{ c }
      \rmat 0
      \\ \hdashline[2.25pt/2.25pt]
      \Imat
    \end{array}\right]
  .
\end{equation}
\egroup

\clearpage
