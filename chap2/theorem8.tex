\section{Állapottér modell}

\begin{about}
  Ismertesse az állapottér modell általános alakját, valamint mutassa be a
  különféle kanonikus alakok átviteli függvény alapján történő származtatását
  (diagonális kanonikus alak, irányíthatósági kanonikus alak, megfigyelhetőségi
  kanonikus alak)
\end{about}

\subsection{A modell általános alakja}

\begin{figure}[htb]
  \centering
  \includestandalone{../graphics/state-space-flow}
  \caption{Az állapottér modell hatásvázlata}
  \label{fig:state-space}
\end{figure}

Eddig a rendszereinket frekvencia tartománybeli átviteli függvényekkel
jellemeztük. Az $Y(s) = W(s) U(s)$ egyenletet időtartományban egy $n$-ed
rendű differenciálegyenlet reprezentálja:
\begin{equation}
  a_{n  } y^{(n)  }(t)
  +
  % a_{n-1} y^{(n-1)}(t)
  % +
  \dots
  +
  a_1 y'(t)
  +
  a_0 y(t)
  =
  b_{r  } u^{(r)  }(t)
  +
  % b_{r-1} u^{(r-1)}(t)
  % +
  \dots
  +
  b_1 u'(t)
  +
  b_0 u(t)
  .
  \label{eq:YWU-time}
\end{equation}

Egy $n$-ed rendű differenciálegyenlet felírható $n$ darab elsőrendű
differenciálegyenlettel. Ez maga az állapottér modell:
\begin{alignat}{9}
   & \dot{\rvec x} (t)       & =                         & \rmat A \rvec x(t) & + & \rmat B \rvec u(t)
   & \quad \rightarrow \quad & \text{állapotegyenlet,}
  \\
   & \rvec y (t)             & =                         & \rmat C \rvec x(t) & + & \rmat D \rvec u(t)
   & \quad \rightarrow \quad & \text{kimeneti egyenlet.}
\end{alignat}

Az állapottér modell felírásához bevezetjük az \textbf{állapot} fogalmát:
az állapot a múlt összesített hatása, bármely időpillanatban leírja a rendszer
viselkedését. Az \textbf{állapotváltozó} a rendszer egyértelmű leírásához
szükséges minimális számú változók vektora.

Az állapottér modellben szereplő vektorok és mátrixok dimenziója:
\bgroup
\def\arraystretch{1.2}
\begin{center}
  \begin{tabular}{| c | c |}
    \hline
    \begin{tabular}{ c c }
      $\rvec x(t) \in \mathbb R^n$ & {állapotvektor}   \\
      $\rvec u(t) \in \mathbb R^m$ & {bemeneti vektor} \\
      $\rvec y(t) \in \mathbb R^p$ & {kimeneti vektor} \\
    \end{tabular}
     &
    \begin{tabular}{ c c }
      $\rmat A \in \mathbb{R}^{n \times n}$ & {rendszermátrix}      \\
      $\rmat B \in \mathbb{R}^{n \times m}$ & {bemeneti mátrix}     \\
      $\rmat C \in \mathbb{R}^{p \times n}$ & {kimeneti mátrix}     \\
      $\rmat D \in \mathbb{R}^{p \times m}$ & {előrecsatoló mátrix} \\
    \end{tabular}
    \\\hline
  \end{tabular}
\end{center}
\egroup

\subsection{Kanonikus alakok származtatása}

\paragraph{Diagonális kanonikus alak}

A diagonális kanonikus alak kialakításánál célunk, hogy az $\rmat A$
\textbf{rendszermátrix diagonális} legyen. Vegyünk példának egy kéttárolós
átviteli függvényt:
\begin{equation}
  W(s) = \frac{K}{(s - p_1)(s- p_2)}.
\end{equation}
Bontsuk parciális törtekké a kifejezést:
\begin{equation}
  W(s) = \frac{A}{s - p_1} + \frac{B}{s - p_2}.
\end{equation}
Az átviteli függvény a bemenet és a kimenet kapcsolatát adja meg. Írjuk fel
tehát a kimeneti egyenletet $s$ tartományban, majd definiáljunk ennek
segítségével két állapotváltozót:
\begin{equation}
  Y(s)
  = W(s) U(s)
  = A \left( \frac{U(s)}{s - p_1} \right)
  + B \left( \frac{U(s)}{s - p_2} \right)
  := A X_1(s) + B X_2(s)
  .
\end{equation}
Hozzuk létre az állapot-egyenleteket is operátor, majd idő tartományban:
\begin{gather}
  X_1(s) = \frac{U(s)}{s - p_1}
  \quad \rightarrow \quad
  s X_2(s) = U(s) + p_1 X(s)
  \quad \rightarrow \quad
  \dot x_1(t) = u(t) + p_1 x_1(t)
  \\
  X_2(s) = \frac{U(s)}{s - p_2}
  \quad \rightarrow \quad
  s X_2(s) = U(s) + p_2 X(s)
  \quad \rightarrow \quad
  \dot x_2(t) = u(t) + p_2 x_2(t)
\end{gather}
Az állapottér egyenletei ezen információk alapján:
\begin{equation}
  \dot{\rvec x}(t) = \begin{bmatrix}
    p_1 & 0 \\ 0 & p_2
  \end{bmatrix} \rvec x(t) + \begin{bmatrix}
    1 \\ 1
  \end{bmatrix} u(t),
  \quad \text{ és } \quad
  y(t) = \begin{bmatrix}
    A & B
  \end{bmatrix} \rvec x(t).
\end{equation}

\paragraph{Irányíthatósági kanonikus alak}

Az irányíthatósági alak származtatásához induljunk ki az alábbi átviteli
függvényből:
\begin{equation}
  W(s) = \frac{A \omega_n^2}{s^2 + 2\zeta\omega_n s + \omega_n^2}.
\end{equation}
Írjuk fel a kimeneti egyenletet, majd definiáljuk az állapotváltozókat:
\begin{equation}
  Y(s)
  = W(s) U(s)
  = A \omega_n^2 \left(\frac{U(s)}{s^2 + 2 \zeta \omega_n s + \omega_n^2}\right)
  := A \omega_n^2 X(s).
\end{equation}
Az állapotegyenlet operátor és idő tartományban:
\begin{equation}
  X(s)
  = \frac{U(s)}{s^2 + 2 \zeta \omega_n s + \omega_n^2}
  \quad \rightarrow \quad
  % s^2 X(s) = - 2 \zeta \omega_n s X(s) - \omega_n^2 X(s) + U(s)
  % \quad \rightarrow \quad
  \ddot x(t) = -2 \zeta \omega_n \dot x(t) - \omega_n^2 x(t) + u(t).
\end{equation}
A második állapotváltozót válasszuk meg az eredeti állapotváltozó deriváltjának,
ekkor az állapottér egyenletei:
\begin{equation}
  \begin{bmatrix}
    \dot x(t) \\ \ddot x(t)
  \end{bmatrix} = \begin{bmatrix}
    0           & 1                  \\
    -\omega_n^2 & - 2 \zeta \omega_n
  \end{bmatrix} \begin{bmatrix}
    x(t) \\ \dot x(t)
  \end{bmatrix} + \begin{bmatrix}
    0 \\ 1
  \end{bmatrix} u(t),
  \quad \text{ és } \quad
  y(t) = \begin{bmatrix}
    A \omega_n^2 & 0
  \end{bmatrix} \begin{bmatrix}
    x(t) \\ \dot x(t)
  \end{bmatrix}.
\end{equation}

\paragraph{Megfigyelhetőségi kanonikus alak} A megfigyelhetőségí kanonikus alak
az irányíthatósági kanonikus alakból származtatható az alábbi összefüggések
alapján:
\begin{equation}
  \rmat A_o = \rmat A_c^\T, \quad
  \rmat B_o = \rmat C_c^\T, \quad
  \rmat C_o = \rmat B_c^\T, \quad
  \rmat D_o = \rmat D_c^\T.
\end{equation}

\clearpage
