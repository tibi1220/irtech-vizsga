\section{Állapotirányíthatóság}

\begin{about}
  Folytonos idejű, lineáris, időinvariáns (LTI) rendszerek esetén ismertesse
  az állapot irányíthatóság definícióját, illetve a vizsgálathoz alkalmazott
  Kálmán-féle rangfeltételt! Mutassa be az állapot-visszacsatolás tervezésének
  lépéseit egy bemenetű egy kimenetű (SISO) rendszerekre nézve hasonlósági
  transzformáció alkalmazása esetén. Válaszában térjen ki az Ackermann-formula
  alkalmazására is!
\end{about}

\subsection{LTI rendszerek állapotirányíthatósága}

\paragraph{Állapotirányíthatóság}

Az adott $\rvec x(t)$ állapot irányítható, ha létezik egy olyan $\rvec u(t)$
bemenet, amely véges $T$ időn belül az állapottér origójába juttatja az
állapotokat, azaz $\rvec x(T) = \nvec$

\paragraph{Kálmán-féle rangfeltétel}

Egy rendszer, melynek dimenziója $n$ akkor állapotirányítható, ha az $\rmat M_c$
irányíthatósági mátrix rangja megegyezik az állapotmátrix dimenziójával, vagyis:
\begin{gather}
  \rank \rmat M_c = \dim \rmat A = n
  \quad - \quad \text{MIMO},
  \\
  \det \rmat M_c \neq 0
  \quad - \quad \text{SISO}.
\end{gather}

Az irányíthatósági mátrix az alábbi alakban írható fel:
\bgroup
\ADLdrawingmode{1}
\def\arraystretch{1.2}
\begin{equation}
  \rmat M_c = \left[\begin{array}{ c : c : c : c }
      \rmat B         &
      \rmat A \rmat B &
      \dots           &
      \rmat A^{n-1} \rmat B
    \end{array}\right]
  .
  \label{eq:M_c}
\end{equation}
\egroup

\subsection{Állapot-visszacsatolás tervezése}

\begin{figure}[htb]
  \centering
  \includestandalone{../graphics/state-feedback}
  \caption{Az állapot-visszacsatolás hatásvázlata}
  \label{fig:state-feedback}
\end{figure}

A tervezés lépései SISO rendszerek esetén:
\begin{enumerate}[label={\color{darkRed}\theenumi})]
  \setcounter{enumi}{-1}
  \item az irányíthatóság vizsgálata ($\det \rmat M_c \overset{?}{\neq} 0$),
  \item irányíthatósági kanonikus alakra hozás (hasonlósági transzformációval),
  \item $\tilde{\rmat K}$ mátrix meghatározása az adott $\lambda_i$ sajátértékek
        segítségével ($\tilde{\tmat A} = \tmat A - \tmat B \tmat K$),
  \item $\tilde{\rmat K}$ mátrix visszatranszformálása $\rmat K$ mátrixszá.
\end{enumerate}

Adott egy állapottér modell állapot-visszacsatolással, melynek egyenletei
($\tvec r(t) \equiv \nvec$):
\begin{alignat}{9}
  \dot{\rvec x(t)} & = \rmat A \rvec x(t) + \rmat B \rvec u(t) \quad \quad
                   &
                   & \text{állapotegyenlet,}
  \\
  \rvec y(t)       & = \rmat C \rvec x(t)
                   &
                   & \text{kimeneti egyenlet,}
  \\
  \rvec u(t)       & = - \rmat K \rvec x(t)
                   &
                   & \text{szabályozó egyenlete.}
\end{alignat}

Vezessük be $\rmat T: \mathbb R^n \mapsto \mathbb R^n$ transzformációs mátrixot
($\rmat T \in \mathbb R^{n \times n}$, invertálható), majd hattassuk ezen
transzformációt az előző egyenletekre ($\tvec x(t) = \rmat T \rvec x(t)$):
\begin{alignat}{9}
  \dot{\tvec x}(t) & = (\rmat T \rmat A \rmat T^{-1}) \tvec x(t) + (\rmat T \rmat B) \rvec u(t)
                   &
                   & = \tmat A \tvec x(t) + \tmat C \rvec u(t),
  \\
  \rvec y(t)       & = (\rmat C \rmat T^{-1}) \tvec x(t)
                   &
                   & = \tmat C \tvec x(t),
  \\
  \rvec u(t)       & = - (\rmat K \rmat T^{-1}) \tvec x(t)
                   &
                   & = - \tmat K \tvec x(t).
\end{alignat}

$\rmat T$ transzformációs mátrix meghatározható az irányíthatósági mátrix
segítségével. Tudjuk hogy az eredeti az alábbi alakban írható fel:
\bgroup
\ADLdrawingmode{1}
\def\arraystretch{1.2}
\begin{equation}
  \rmat M_c = \left[\begin{array}{ c : c : c : c }
      \rmat B         &
      \rmat A \rmat B &
      \dots           &
      \rmat A^{n-1} \rmat B
    \end{array}\right]
  .
  \label{eq:M_c-og}
\end{equation}
\egroup

A transzformált irányíthatósági mátrix ezek alapján:
\bgroup
\ADLdrawingmode{1}
\def\arraystretch{1.2}
\begin{alignat}{9}
  \tmat M_c
   & =         &  & \left[\begin{array}{ c : c : c : c }
                              \tmat B         &
                              \tmat A \tmat B &
                              \dots           &
                              \tmat A^{n-1} \tmat B
                            \end{array}\right],
  \\
  \tmat M_c
   & =         &  & \left[\begin{array}{ c : c : c : c }
                              (\rmat T \rmat B)                                &
                              (\rmat T \rmat A \rmat T^{-1}) (\rmat T \rmat B) &
                              \dots                                            &
                              (\rmat T \rmat A \rmat T^{-1})^{n-1} (\rmat T \rmat B)
                            \end{array}\right],
  \\
  \tmat M_c
   & = \rmat T &  & \left[\begin{array}{ c : c : c : c }
                              \rmat B         &
                              \rmat A \rmat B &
                              \dots           &
                              \rmat A^{n-1} \rmat B
                            \end{array}\right],
  \\
  \tmat M_c
   & = \rmat T &  & \rmat M_c.
\end{alignat}
\egroup

Ezek alapján a transzformációs mátrix az alábbi képlettel határozható meg:
\begin{equation}
  \rmat T = \tmat M_c \rmat M_c^{-1}.
  \label{eq:T-calc}
\end{equation}

Az állapotvektort és a visszacsatolási mátrixot ezen transzformációs mátrix
ismeretében már vissza tudjuk transzformálni a kívánt alakra:
\begin{equation}
  \rvec x(t) = \rmat T^{-1} \tvec x(t),
  \quad \text{ és } \quad
  \rmat K = \tmat K \rmat T.
  \label{eq:T-back}
\end{equation}

\paragraph{Ackermann-formula}

$\rmat K$ visszacsatolási mátrix az alábbi képlettel is meghatározható:
\begin{equation}
  \rmat K
  = \uvec e_n^\T
  \;\rmat M_c^{-1}
  \;\varphi(\rmat A)
  .
  \label{eq:Acker-K}
\end{equation}
A képletben szereplő változók elnevezése:
\bgroup
\def\arraystretch{1.2}
\begin{center}
  \begin{tabular}{ | c c | }
    \hline
    $\rmat A, \rmat K$ & rendszer- és visszacsatolási mátrix  \\
    $\uvec e_n^\T$     & $n$-edik egységvektor transzponáltja \\
    $\rmat M_c^{-1}$   & az irányíthatósági mátrix inverze    \\
    $\varphi(\rmat X)$ & elvárt pólusokat tartalmazó polinom  \\
    \hline
  \end{tabular}
\end{center}
\egroup

\paragraph{Példa} Legyen egy $m$ tömegünk, melyet $f$ erővel gerjesztünk, a
kimenet az elmozdulás, tehát $y = x$, valamint Newton törvénye alapján
$f = ma = m \ddot x$. Az Állapottér-modell egyenletei ekkor:
\begin{equation}
  \begin{bmatrix}
    \dot x(t) \\ \ddot x(t)
  \end{bmatrix} = \begin{bmatrix}
    0 & 1 \\ 0 & 0
  \end{bmatrix} \begin{bmatrix}
    x(t) \\ \dot x(t)
  \end{bmatrix} + \begin{bmatrix}
    0 \\ \slashfrac{1}{m}
  \end{bmatrix} u(t),
  \quad \text{ és } \quad
  y(t) = \begin{bmatrix}
    1 & 0
  \end{bmatrix} \begin{bmatrix}
    x(t) \\ \dot x(t)
  \end{bmatrix}.
\end{equation}
Az irányíthatósági mátrix ezek alapján:
\begin{equation}
  \rmat M_c = \begin{bmatrix}
    \rmat B & \rmat A \rmat B
  \end{bmatrix} = \begin{bmatrix}
    0                & \slashfrac{1}{m} \\
    \slashfrac{1}{m} & 0
  \end{bmatrix} = \frac{1}{m} \begin{bmatrix}
    0 & 1 \\ 1 & 0
  \end{bmatrix},
\end{equation}
melynek a determinánsa nem zérus (hanem $-1/m^2$), tehát nem szinguláris,
a rendszer irányítható. A mátrix inverze:
\begin{equation}
  \rmat M_c^{-1} = \frac{1}{-1/m^2} \frac{1}{m} \begin{bmatrix}
    0 & -1 \\ -1 & 0
  \end{bmatrix} = \begin{bmatrix}
    0 & m \\ m & 0
  \end{bmatrix} = m \begin{bmatrix}
    0 & 1 \\ 1 & 0
  \end{bmatrix}.
\end{equation}
Írjuk fel a kimenet és bemenet közötti kapcsolatot Laplace tartományban:
\begin{equation}
  Y(s) = \frac{1 / m}{s^2} U(s),
\end{equation}
Vezessünk be egy új állapotot:
\begin{equation}
  X(s):=\frac{1}{s^2} U(s)
  \quad \rightarrow \quad
  Y(s) = \frac{1}{m} X(s).
\end{equation}
Ezen egyenletek alapján az állapottér modell:
\begin{equation}
  \begin{bmatrix}
    \dot x(t) \\ \ddot x(t)
  \end{bmatrix} = \begin{bmatrix}
    0 & 1 \\ 0 & 0
  \end{bmatrix} \begin{bmatrix}
    x(t) \\ \dot x(t)
  \end{bmatrix} + \begin{bmatrix}
    0 \\ 1
  \end{bmatrix} u(t),
  \quad \text{ és } \quad
  y(t) = \begin{bmatrix}
    \slashfrac{1}{m} & 0
  \end{bmatrix} \begin{bmatrix}
    x(t) \\ \dot x(t)
  \end{bmatrix}.
\end{equation}
A módisított irányíthatósági mátrix a módosított állapotegyenlet alapján:
\begin{equation}
  \tmat M_c = \begin{bmatrix}
    \tmat B & \tmat A \tmat B
  \end{bmatrix} = \begin{bmatrix}
    0 & 1 \\
    1 & 0
  \end{bmatrix}
\end{equation}
$\rmat T$ transzformációs mátrix számításához minden adat rendelkezésre áll:
\begin{equation}
  \rmat T
  = \tmat M_c \rmat M_c^{-1}
  = \begin{bmatrix}
    0 & 1 \\ 1 & 0
  \end{bmatrix} \begin{bmatrix}
    0 & m \\ m & 0
  \end{bmatrix} = \begin{bmatrix}
    m & 0 \\ 0 & m
  \end{bmatrix} = m \begin{bmatrix}
    1 & 0 \\ 0 & 1
  \end{bmatrix}.
\end{equation}
Allokáljunk új pólusokat, ezek paraméteresen: $\tilde p_1$ és $\tilde p_2$,
ezek alapján a karakterisztikus polinom:
\begin{equation}
  \varphi(s)
  = (s - \tilde p_1)(s - \tilde p_2)
  = s^2
  + s (-\tilde p_1 - \tilde p_2)
  + (\tilde p_1 \tilde p_2).
\end{equation}
A rendszermátrix a pólusok alapján:
\begin{equation}
  \tilde{\tmat A} = \begin{bmatrix}
    0                       & 1                       \\
    - \tilde p_1 \tilde p_2 & \tilde p_1 + \tilde p_2
  \end{bmatrix}.
\end{equation}
A módosított rendszermátrix együtthatói az alábbi egyenlettel számítható:
\begin{gather}
  \tilde{\tmat A} = \tmat A - \tmat B \tmat K,
  \\
  \begin{bmatrix}
    0                       & 1                       \\
    - \tilde p_1 \tilde p_2 & \tilde p_1 + \tilde p_2
  \end{bmatrix} = \begin{bmatrix}
    0 & 1 \\ 0 & 0
  \end{bmatrix} - \begin{bmatrix}
    0 \\ 1
  \end{bmatrix} \begin{bmatrix}
    \tilde k_1 & \tilde k_2
  \end{bmatrix},
  \\
  \tmat K = \begin{bmatrix}
    \tilde k_1 \\
    \tilde k_2
  \end{bmatrix}^\T = \begin{bmatrix}
    \tilde p_1 \tilde p_2 \\
    -(\tilde p_1 + \tilde p_2)
  \end{bmatrix}^\T.
\end{gather}
Ezek alapján a módosítatlan rendszermátrix:
\begin{equation}
  \rmat K
  = \tmat K \rmat T
  = \begin{bmatrix}
    \tilde k_1 & \tilde k_2
  \end{bmatrix} \begin{bmatrix}
    m & 0 \\ 0 & m
  \end{bmatrix} = \begin{bmatrix}
    m \tilde k_1 \\
    m \tilde k_2
  \end{bmatrix}^\T = \begin{bmatrix}
    m (\tilde p_1 \tilde p_2) \\
    -m (\tilde p_1 + \tilde p_2)
  \end{bmatrix}^\T.
\end{equation}
$\rmat K$ mátrixot az Ackermann-formula segítségével is meg tudjuk határozni:
\begin{gather}
  \rmat K = \begin{bmatrix}
    0 \\ 1
  \end{bmatrix}^\T \begin{bmatrix}
    0 & m \\ m & 0
  \end{bmatrix} \left(
  \begin{bmatrix}
    0 & 1 \\ 0 & 0
  \end{bmatrix}^2 - (\tilde p_1 + \tilde p_2) \begin{bmatrix}
    0 & 1 \\ 0 & 0
  \end{bmatrix} + \tilde p_1 \tilde p_2 \begin{bmatrix}
    1 & 0 \\ 0 & 1
  \end{bmatrix}
  \right),
  \\
  \rmat K = \begin{bmatrix}
    m \\ 0
  \end{bmatrix}^\T \begin{bmatrix}
    \tilde p_1 \tilde p_2 & -(\tilde p_1 + \tilde p_2) \\
    0                     & \tilde p_1 \tilde p_2
  \end{bmatrix} = \begin{bmatrix}
    m (\tilde p_1 \tilde p_2) \\
    -m (\tilde p_1 + \tilde p_2)
  \end{bmatrix}^\T.
\end{gather}
Látható, hogy bármely módszert használva ugyanarra az eredményre jutottunk.

